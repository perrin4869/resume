% LaTeX file for resume 
% This file uses the resume document class (res.cls)

\documentclass{res} 
%\usepackage{helvetica} % uses helvetica postscript font (download helvetica.sty)
%\usepackage{newcent}   % uses new century schoolbook postscript font 
\newsectionwidth{0pt}  % So the text is not indented under section headings
\usepackage{fancyhdr}  % use this package to get a 2 line header
\usepackage{hyperref}
\renewcommand{\headrulewidth}{0pt} % suppress line drawn by default by fancyhdr
\setlength{\headheight}{24pt} % allow room for 2-line header
\setlength{\headsep}{24pt}  % space between header and text
\setlength{\headheight}{24pt} % allow room for 2-line header
\pagestyle{fancy}     % set pagestyle for document
% \rhead{ {\it Z. Zinger}\\{\it p. \thepage} } % put text in header (right side)
\cfoot{}                                     % the foot is empty
\topmargin=-0.5in % start text higher on the page

\hypersetup{
  colorlinks=true,
  linkcolor=blue,
  filecolor=magenta,
  urlcolor=cyan,
}

\begin{document}
\thispagestyle{empty} % this page has no header  
\name{JULIAN GRINBLAT\\[12pt]} % the \\[12pt] adds a blank line after name

\address{\begin{tabular}{l}{\bf 住所} \\ % for some reason since the second address is a table, the first one has to be too to get the alignment correctly
   146-0085, 東京都, 大田区  \\
   久が原 3-30-9, コリーヌ久が原104号室 \\
   (090) 7768-6902
\end{tabular}}

\address{\begin{tabular}{@{}ll}
   Email: & \href{mailto:julian@dotcore.co.il}{julian@dotcore.co.il} \\
   GitHub: & \url{https://github.com/perrin4869} \\
   NPM: & \url{https://www.npmjs.com/~perrin4869}
\end{tabular}}

\begin{resume}
 
\section{\centerline{職歴}} 
\dates{2019年2月 - 現在}
\vspace{8pt}
{\sl デロイトトーマツコンサルティング合同会社} \hfill        2019年2月 - 現在
\vspace{8pt}

\title{SOFTWARE ENGINEER}
\employer{}
\location{}
\begin{position}
ストラテジーユニットでTechHarborプロジェクトの開発を務めた。
主な責任はreactベースのウェブアプリとserverlessベースのgraphql APIの開発、メインテナンス、及びデプロイ.\\

 \begin{itemize} \itemsep -2pt % reduce space between items
  \item Oversee development and integration of new features by remote oversea developers.
    Review all pull requests and provide guidance towards the proferred approach.
  \item Migrate from AWS Cognito based OAuth 2.0 implicit grant type to Authorization Code grant type
    to gain advantage of refresh tokens and remove 1 hour login limit.
  \item react-apolloの公式フックAPIを利用して、apollo-lambdaベースのgraphql APIからデータのフェッチ
    を実現。Identity tokenとRefresh tokenを正しくインテグレートした。フックのユニットテストを書いた。
  \item Utilize react-apollo based react hooks to fetch data from our apollo-lambda based API.
    Correcly manage application state and cache for performance. Integrate with identity tokens.
    Author unit tests for all hooks.
  \item Deploy a DynamoDB instance to store user-defined searches and relevant information.
    Expose API through graphql queries and mutations and write relevant unit and integration tests.
    Correctly display results on the client side, including form state managed by react-final-form,
    search results with react-apollo and blueprintjs tables, infinite scrolling based pagination.
  \item Store data exports over 5MB in S3, which are over the limit of what AWS Lambda
    functions can return. Respond with an HTTP redirect, handle in the client and save using file-saver.
  \item Setup docker-compose based development environment with elasticsearch, minio, and DynamoDB.
    Author custom docker images to create DynamoDB data tables and minio buckets. Run integration tests
    against this environment.
  \item Expose documents to registered users of our application via an S3 bucket, and manage access to objects
    in the bucket through a Cognito Identity Pool, our Cognito user pool, and the Javascript aws-sdk.
  \item Provide user behavior analytics, such as routes visited, sign in, sign out, session length, user retention,
    application errors, through an Amazon Pinpoint deployment.
  \item Implement recharts based graphs to display search data aggregations. Author relevant graphql queries.
  \item Utilize react-spring and react-transition-group to provide transition and scrolling animations.
 \end{itemize}
\end{position}

\vspace{8pt}
{\sl 楽天株式会社} \hfill        2015年4月~2019年1月
\vspace{8pt}

\title{APPLICATION ENGINEER}
\employer{}
\location{}
\dates{2018年1月~2019年1月}
\begin{position}
セルフサービス環境のフロントエンドポータル開発を務めた。本来、
このセルフサービス環境は、Reactコンポネントのみとして
開発されたものであったが、多様なライブラリーを導入することによって、
コードの構成を整理し、テストを可能にした。\\

 \begin{itemize} \itemsep -2pt % reduce space between items
  \item Reduxを利用して、アプリケーションのGlobal Stateを管理した。
        ミドルウェアを使って、APIから情報を読み取り、
        Local Storageでのデータ保存等の副作用を可能にした。
        reselectとre-reselectでセレクターを書いた。
  \item eslintとairbnbのルールを用いてコードのlintingを実現した。
  \item mocha、chai、sinon、enzymeに基づいたテストを構築した。
  \item react-final-formを導入して、フォームのstate管理を向上させた。
  \item PostCSSを実行することで、ポータルのCSSのビルドを達成した。
        これによってBootstrapやFontAwesome、CSSライブラリーの管理を
        より簡単にできた。
  \item webpack と imagemin で画像の圧縮を実現した。
 \end{itemize}
\end{position}

\title{DEVOPS ENGINEER}
\employer{}
\location{}
\dates{2015年6月~2017年12月}
\begin{position}
入社後の新卒トレーニングを経て、サーバープラットフォームチームに配属された。
様々なストレージソリュション(Pure Storage, Nimble Storage等)や、OpenStack環境、
NuageのSDN環境の管理を行った。
その際、達成したタスクは以下の通りである:\\

\begin{itemize} \itemsep -2pt % reduce space between items
  \item Fluentd、InfluxDB、Grafana、SNMP、ParamikoやPythonのスクリプトを用いて
        ストレージソリューションのレーテンシーやIOPS、テレメトリデータのダッシュボード
        を構築した。
  \item Python スクリプトで SDN オペレーションの自動化を実現した。
        バックアップや VMware Cluster の資源の配分、常に行うオペレーションは
        より簡単かつ正確になった。VMware を pyVmomi で操作した。
  \item OpenStack環境でsysctl変数はリブート後に消えるという問題を解決した。
        Ansible の playbook を適切に編集して、必要なノードに変更を施した。
  \item OpenStack環境のVMをテナント交換した。MySQLレコードを手動で変更するやりかたを
        調べ、テスト環境で試した。マニュアルを書いて、オペレーションを無事に実行した。
\end{itemize}
\end{position}

\vspace{0.2in} 
\section{\centerline{学歴}} 
\vspace{8pt} 
{\sl 学士(理学)}, 物理 \\ % \sl will be bold italic in
大阪大学 理学部物理学科      \hfill    2015年3月
  
\vspace{0.2in} 

\section{\centerline{コース}}
\vspace{8pt} 

\begin{tabular}{@{}ll}
  M101: MongoDB for Developers & \href{http://university.mongodb.com/course_completion/066efa7d13f84df789a4ddff81524640}{Certificate} \\
  M102: MongoDB for DBAs & \href{http://university.mongodb.com/course_completion/797324e4c82e4a27b2d33c3749476003}{Certificate} \\
  Kubernetes Certification by Mirantis (KCM100) & \href{https://training.mirantis.com/verify/certificate/status/bF19JZmlQoYyAoU4jkY15w/100-457-704}{Certificate} \\
\end{tabular}


\newpage

\section{\centerline{オープンソース コントリビューション}}
\vspace{8pt} 

JavaScriptやNode.jsを中心に、オープンソースソフトウェアに
コントリビューションをした。これらは、GitHubで一般に公開されている。

\subsubsection{プルリクエスト}

\begin{itemize}
  \item \url{https://github.com/socketio/socket.io/pull/2745} \\
  上記のリンクは、バージョン1.5.0でquery stringの変数が正確に伝わっていなかった
  というバッグ解除のPRである。ここでは、ES5のObject.assignを使って伝えた。
  対応中の旧IEでは、polyfillを使うのが、ポイントである。

  \item \url{https://github.com/socketio/engine.io/pull/444} \\
  バージョン1.5.1のバッグ解除のPRである。回帰テストを追加した。

  \item \url{https://github.com/reduxjs/react-redux/pull/1288} \\
  新しいreact-reduxのフックAPIにおいて、コンポーネントのレンダリングの仕組みを
  リファレンスの比較で決めるようにした。useState等、公式のReactのフックと同じ仕組みである。

  \item \url{https://github.com/agentk/fontfacegen/pull/26} \\
  これは、FontForgeというソフトウェアを使って、ウェブ用のフォントを出力するモジュールである。
  このPRは、出力されるフォントの文字選択機能を追加するものである。
  日本語のようにフォントのサイズは複数のMBにもなりうる言語の場合では、
  利用する文字だけを限定して選択することを可能にすることで、
  フォント・サイズの縮小がより簡単となる。

  \item \url{https://github.com/libxmljs/libxmljs/pull/521} \\
  XML の processing instructions に対応する機能追加の PR である。
  svg ファイルにスタイルシートを追加するのに便利である。
  libxmljs は裏で C 言語の libxml2 が主に動作しているので、PR の内容の
  殆どは C++のコードで書かれている。このPRを実施することによって、
  Node.JS の JavaScript エンジンの V8の仕組みを理解するのに極めて役立った。

  \item \url{https://github.com/final-form/react-final-form/pull/266} \\
  Observer パターンに基づいた、Form State を管理する React コンポネント集である。
  Lookahead コンポネントを react-final-form と redux で使う例を追加した PR である。
  final-form の様々な機能が実例されている。

  \item \url{https://github.com/jasonpincin/stream-to-string/pull/1} \\
  Node のストリームを文字列に変換するモジュールである。
  変換するときの encoding を指定できるよう、新しいオプションを加えた。
  回帰テストも追加した。PR の内容は、自分の NPM モジュール、vinyl-to-string
  の中にも活用した。

  \item \url{https://github.com/izolate/html2pug/pull/2} \\
  HTMLを、Expressアプリケーションで一般的に用いられるPugへ変換するモジュールである。
  この PR の目的は、HTML Parser を jsdom から parse5 に変えることで、
  HTML のコメントや HTML Fragment 等というより多くの機能に対応させることであった。

  \item \url{https://github.com/itgalaxy/favicons/pull/110} \\
  favicon と関連したファイルをプラットフォーム毎に作成するためのモジュールである。
  この PR は、Gulp で使う場合、出力される HTML をパイプラインに入れる機能を
  追加したものである。これによって、gulp-html2pug等といった他の
  Gulp プラグインでの HTML に対するさらなる変更が可能となった。

\end{itemize}

\subsubsection{NPM モジュール}

\begin{itemize}
  \item \url{https://www.npmjs.com/package/i18next-conv} \\
  このモジュールの管理人になることができた。
  gettext の PO ファイルを i18next の JSON フォーマットに変換する
  モジュールである。
  これは一週間当りに 5000 回以上ダウンロードされている。

  \item \url{https://www.npmjs.com/package/i18next-fetch-backend} \\
  普段 i18next で利用される XHR バックエンドの代わりに利用できる。
  これはfetch を利用しているため、クライアントとサーバー両方が
  i18nextコードを扱うの際には特に役立つ。
  一ヶ月当り 2000 回以上ダウンロードされている。

  \item \url{https://www.npmjs.com/package/react-scroll-ondrag} \\
  マウスのドラッグで、Reactのエレメントにスクロールの機能を可能にする
  フックのモジュール。

  \item \url{https://www.npmjs.com/package/react-stay-scrolled} \\
  チャットウィンドウに新しいメッセージが現れるなどの場合において、
  スクロールダウンするのに役立つ React コンポネントである。
  GitHub で 星を34も獲得しており、一ヶ月当り1000 回以上ダウンロードされている。

  \item \url{https://www.npmjs.com/package/redux-persistent} \\
  Meant as an alternative to redux-persist, providing a selector based API,
  while remaining compatible with its storage ecosystem. Persists redux state
  inside a permanent storage, such as localStorage, IndexDB, remote storage,
  and hydrates said state back into the store.


  \item \url{https://www.npmjs.com/package/vinyl-contents-tostring} \\
  vinylファイル(Gulpが使うバーチャルファイル)のコンテンツを文字列に変換する
  モジュールである。自作のGulpプラグインが利用している。
  一ヶ月当り 1000 回以上ダウンロードされている。

  \item \url{https://www.npmjs.com/package/rollup-plugin-i18next-conv} \\
  Rollupでgettextファイルをi18nextのフォーマットでインポートできるモジュール。
  webviewのアプリケーション等、訳がオフラインでも必要な場合に便利である。

  \item \url{https://www.npmjs.com/package/gulp-xml-transformer} \\
  Gulp で XML 編集モジュールである。
  一ヶ月当り1000 回以上ダウンロードされている。
\end{itemize}

\vspace{0.2in} 
\section{\centerline{言語}} 
\vspace{15pt}
\begin{itemize}
   \item スペイン語、ネイティブ
   \item 英語、流暢
   \item ヘブライ語、流暢
   \item 日本語、 日本語能力試験一級合格
\end{itemize}


\vspace{0.2in} 
 
\end{resume} 
\end{document}
