% LaTeX file for resume 
% This file uses the resume document class (res.cls)

\documentclass{res} 
%\usepackage{helvetica} % uses helvetica postscript font (download helvetica.sty)
%\usepackage{newcent}   % uses new century schoolbook postscript font 
\newsectionwidth{0pt}  % So the text is not indented under section headings
\usepackage{fancyhdr}  % use this package to get a 2 line header
\usepackage{hyperref}
\renewcommand{\headrulewidth}{0pt} % suppress line drawn by default by fancyhdr
\setlength{\headheight}{24pt} % allow room for 2-line header
\setlength{\headsep}{24pt}  % space between header and text
\setlength{\headheight}{24pt} % allow room for 2-line header
\pagestyle{fancy}     % set pagestyle for document
% \rhead{ {\it Z. Zinger}\\{\it p. \thepage} } % put text in header (right side)
\cfoot{}                                     % the foot is empty
\topmargin=-0.5in % start text higher on the page

\hypersetup{
  colorlinks=true,
  linkcolor=blue,
  filecolor=magenta,
  urlcolor=cyan,
}

\begin{document}
\thispagestyle{empty} % this page has no header  
\name{JULIAN GRINBLAT\\[12pt]} % the \\[12pt] adds a blank line after name

\address{\begin{tabular}{l}{\bf 住所} \\ % for some reason since the second address is a table, the first one has to be too to get the alignment correctly
   146-0085, 東京都, 大田区  \\
   久が原 3-30-9, コリーヌ久が原104号室 \\
   (090) 7768-6902
\end{tabular}}

\address{\begin{tabular}{@{}ll}
   Email: & \href{mailto:julian@dotcore.co.il}{julian@dotcore.co.il} \\
   GitHub: & \url{https://github.com/perrin4869} \\
   NPM: & \url{https://www.npmjs.com/~perrin4869}
\end{tabular}}

\begin{resume}
 
\section{\centerline{職歴}} 
\vspace{8pt}
{\sl 楽天株式会社} \hfill        2015年4月~現在
\vspace{8pt}

\title{APPLICATION ENGINEER}
\employer{}
\location{}
\dates{2018年1月~現在}
\begin{position}
セルフサービス環境のフロントエンドポータル開発を務めた。本来、
このセルフサービス環境は、Reactコンポネントのみとして
開発されたものであったが、多様なライブラリーを導入することによって、
コードの構成を整理し、テストを可能にした。
使用したライブラリは、redux、react-final-form、ramda、normalizr,
reselect、re-reselect、mocha、chai、sinon、postcss等である。\\

 \begin{itemize} \itemsep -2pt % reduce space between items
  \item Reduxを利用して、アプリケーションのGlobal Stateを管理した。
        ミドルウェアを使って、APIから情報を読み取り、
        Local Storageでのデータ保存等の副作用を可能にした。
        reselectとre-reselectでセレクターを書いた。
  \item eslintとairbnbのルールを用いてコードのlintingを実現した。
  \item mocha、chai、sinon、enzymeに基づいたテストを構築した。
  \item react-final-formを導入して、フォームのstate管理を向上させた。
  \item PostCSSを実行することで、ポータルのCSSのビルドを達成した。
        これによってBootstrapやFontAwesome、CSSライブラリーの管理を
        より簡単にできた。
  \item webpack と imagemin で画像の圧縮を実現した。
 \end{itemize}
\end{position}

\title{DEVOPS ENGINEER}
\employer{}
\location{}
\dates{2015年6月~2017年12月}
\begin{position}
入社後の新卒トレーニングを経て、サーバープラットフォームチームに配属された。
様々なストレージソリュション(Pure Storage, Nimble Storage等)や、OpenStack環境、
NuageのSDN環境の管理を行った。
その際、達成したタスクは以下の通りである:\\

\begin{itemize} \itemsep -2pt % reduce space between items
  \item Fluentd、InfluxDB、Grafana、SNMP、ParamikoやPythonのスクリプトを用いて
        ストレージソリューションのレーテンシーやIOPS、テレメトリデータのダッシュボード
        を構築した。
  \item Python スクリプトで SDN オペレーションの自動化を実現した。
        バックアップやVMware Clusterの資源の配分、常に行うオペレーションは
        より簡単と正確になった。VMwareをpyVmomiで操作した。
  \item OpenStack環境でsysctl変数はリブート後に消えるという問題を解決した。
        Ansible の playbook を適切に編集して、必要なノードに変更を施した。
  \item OpenStack環境のVMをテナント交換した。MySQLレコードを手動で変更するやりかたを
        調べ、テスト環境で試した。マニュアルを書いて、オペレーションを無事に実行した。
\end{itemize}
\end{position}

\vspace{0.2in} 
\section{\centerline{学歴}} 
\vspace{8pt} 
{\sl 学士(理学)}, 物理 \\ % \sl will be bold italic in
大阪大学 理学部物理学科      \hfill    2015年3月
  
\vspace{0.2in} 

\newpage

\section{\centerline{オープンソース コントリビューション}}
\vspace{8pt} 

JavaScriptやNode.jsを中心に、オープンソースソフトウェアに
コントリビューションをした。これらは、GitHubで一般に公開されている。

\subsubsection{プルリクエスト}

\begin{itemize}
  \item \url{https://github.com/socketio/socket.io/pull/2745} \\
  上記のリンクは、バージョン1.5.0でquery stringの変数が正確に伝わっていなかった
  というバッグ解除のPRである。ここでは、ES5のObject.assignを使って伝えた。
  対応中の旧IEでは、polyfillを使うのが、ポイントである。

  \item \url{https://github.com/socketio/engine.io/pull/444} \\
  バージョン1.5.1のバッグ解除のPRである。回帰テストを追加した。

  \item \url{https://github.com/agentk/fontfacegen/pull/26} \\
  これは、FontForgeというソフトウェアを使って、ウェブ用のフォントを出力するモジュールである。
  このPRは、出力されるフォントの文字選択機能を追加するものである。
  日本語のようにフォントのサイズは複数のMBにもなりうる言語の場合では、
  利用する文字だけを限定して選択することを可能にすることで、
  フォント・サイズの縮小がより簡単となる。

  \item \url{https://github.com/libxmljs/libxmljs/pull/521} \\
  XMLのprocessing instructionsが対応される、機能追加のPRである。
  svgファイルにスタイルシートを追加するのに便利である。
  libxmljs は裏で C 言語の libxml2 が主に動作しているので、PRの内容の
  殆どはC++のコードで書かれている。Node.JSのJavaScriptエンジンのV8の
  働き方のこと、わかるようになることが多かった。

  \item \url{https://github.com/final-form/react-final-form/pull/266} \\
  Observer パターンに基づいた、Form Stateを管理するReactコンポネント集である。
  Lookaheadコンポネントをreact-final-formとreduxで使う例を追加したPRである。
  final-formの様々な機能が実例される。

  \item \url{https://github.com/jasonpincin/stream-to-string/pull/1} \\
  Nodeのストリームを文字列に変換するモジュールである。
  変換するときのencodingを指定できるよう、新しいオプションを入れた。
  回帰テストも追加した。
  PRの内容は、自分のNPMモジュール、vinyl-to-stringで利用した。

  \item \url{https://github.com/izolate/html2pug/pull/2} \\
  Express アプリケーションでで頻繁に利用される Pug を、HTMLから変換するモジュールである。
  このPRの目的は、HTML Parserをjsdomからparse5に変えることで、HTMLのコメント
  やHTML Fragment等、より多くの機能に対応することであった。

  \item \url{https://github.com/itgalaxy/favicons/pull/110} \\
  faviconと関連しているファイルをプラットフォーム毎に作成するためのモジュールである。
  このPRは、Gulpで使う場合、出力されるHTMLをパイプラインに入れてくれる機能を
  追加する。gulp-html2pug等、他のGulpプラグインでHTMLを更に変更可能になった。

  \item \url{https://github.com/JRJurman/rollup-plugin-polyfill/pull/2} \\
  Rollupでビルドするアプリケーションにpolyfillを注入するモジュールである。
  このPRでSource Mapsの作成ができるようになった。ビルド時の警告メッセージが
  消えた。もっとも、CommonJSとESMの2つのフォーマットでモジュールのインポート
  ができるようになった。
\end{itemize}

\subsubsection{NPM モジュール}

\begin{itemize}
  \item \url{https://www.npmjs.com/package/i18next-conv} \\
  このモジュールの管理人になることができた。
  gettext の PO ファイルを i18next の JSON フォーマットに変換する モジュールである。
  一週間で5000回以上にダウンロードされる。

  \item \url{https://www.npmjs.com/package/i18next-fetch-backend} \\
  普段i18nextで利用されるXHRバックエンドの代わりに利用できる。
  fetchを利用するから、クライアントとサーバー両方が扱うコードの場合
  に特に役立つ。
  一ヶ月で2000回以上にダウンロードされる。

  \item \url{https://www.npmjs.com/package/react-stay-scrolled} \\
  チャットウィンドウに新しいメッセージがある等の時に、スクロールダウンするのに
  役立つReactコンポネントである。
  GitHubで34の星があり、一ヶ月で1000回以上にダウンロードされる。

  \item \url{https://www.npmjs.com/package/vinyl-contents-tostring} \\
  vinylファイル(Gulpが使うバーチャルファイル)のコンテンツを文字列に変換する
  モジュールである。自作のGulpプラグインが利用している。
  一ヶ月で1000回以上にダウンロードされる。

  \item \url{https://www.npmjs.com/package/rollup-plugin-i18next-conv} \\
  Rollupでgettextファイルをi18nextのフォーマットでインポートできるモジュール。
  webviewのアプリケーション等、訳がオフラインでも必要な場合に便利である。

  \item \url{https://www.npmjs.com/package/gulp-xml-transformer} \\
  GulpでXML編集モジュールである。
  一ヶ月で1000回以上にダウンロードされる。
\end{itemize}

\vspace{0.2in} 
\section{\centerline{言語}} 
\vspace{15pt}
\begin{itemize}
   \item スペイン語、ネイティブ
   \item 英語、流暢
   \item ヘブライ語、流暢
   \item 日本語、 日本語能力試験一級合格
\end{itemize}


\vspace{0.2in} 
 
\end{resume} 
\end{document}
