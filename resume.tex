% LaTeX file for resume 
% This file uses the resume document class (res.cls)

\documentclass{res} 
%\usepackage{helvetica} % uses helvetica postscript font (download helvetica.sty)
%\usepackage{newcent}   % uses new century schoolbook postscript font 
\newsectionwidth{0pt}  % So the text is not indented under section headings
\usepackage{fancyhdr}  % use this package to get a 2 line header
\usepackage{hyperref}
\renewcommand{\headrulewidth}{0pt} % suppress line drawn by default by fancyhdr
\setlength{\headheight}{24pt} % allow room for 2-line header
\setlength{\headsep}{24pt}  % space between header and text
\setlength{\headheight}{24pt} % allow room for 2-line header
\pagestyle{fancy}     % set pagestyle for document
% \rhead{ {\it Z. Zinger}\\{\it p. \thepage} } % put text in header (right side)
\cfoot{}                                     % the foot is empty
\topmargin=-0.5in % start text higher on the page

\hypersetup{
  colorlinks=true,
  linkcolor=blue,
  filecolor=magenta,
  urlcolor=cyan,
}

\begin{document}
\thispagestyle{empty} % this page has no header  
\name{JULIAN GRINBLAT\\[12pt]} % the \\[12pt] adds a blank line after name

\address{\begin{tabular}{l}{\bf ADDRESS} \\ % for some reason since the second address is a table, the first one has to be too to get the alignment correctly
   146-0085, Tokyo Prefecture, Ota-ku  \\
   Kugahara 3-30-9, Room 104 \\
   (090) 7768-6902
\end{tabular}}

\address{\begin{tabular}{@{}ll}
   Email: & \href{mailto:julian@dotcore.co.il}{julian@dotcore.co.il} \\
   GitHub: & \url{https://github.com/perrin4869} \\
   NPM: & \url{https://www.npmjs.com/~perrin4869}
\end{tabular}}

\begin{resume}
 
\section{\centerline{PROFESSIONAL EXPERIENCE}} 
\vspace{8pt}
{\sl Deloitte Tohmatsu Consulting, Tokyo, Japan} \hfill        February 2019 - Present
\vspace{8pt}

\title{SOFTWARE ENGINEER}
\employer{}
\location{}
\dates{February 2019 - Present}
\begin{position}
Working on the TechHarbor project in the Strategy unit. Main responsibilities are development, maintenance and deployment of the react-based frontend and the graphql API.\\

 \begin{itemize} \itemsep -2pt % reduce space between items
  \item Oversee development and integration of new features by remote oversea developers.
    Review all pull requests and provide guidance towards the preferred approach.
  \item Migrate from AWS Cognito based OAuth 2.0 implicit grant type to Authorization Code grant type
    to gain advantage of refresh tokens and remove 1 hour login limit.
  \item Utilize react-apollo based react hooks to fetch data from our apollo-lambda based API.
    Correcly manage application state and cache for performance. Integrate with identity tokens.
    Author unit tests for all hooks.
  \item Deploy a DynamoDB instance to store user-defined searches and relevant information.
    Expose API through graphql queries and mutations and write relevant unit and integration tests.
    Correctly display results on the client side, including form state managed by react-final-form,
    search results with react-apollo and blueprintjs tables, infinite scrolling based pagination.
  \item Store data exports over 5MB in S3, which are over the limit of what AWS Lambda
    functions can return. Respond with an HTTP redirect, handle in the client and save using file-saver.
  \item Setup docker-compose based development environment with elasticsearch, minio, and DynamoDB.
    Author custom docker images to create DynamoDB data tables and minio buckets. Run integration tests
    against this environment.
  \item Implement recharts based graphs to display search data aggregations. Author relevant graphql queries.
  \item Utilize react-spring and react-transition-group to provide animations.
 \end{itemize}
\end{position}

\vspace{8pt}
{\sl Rakuten, Tokyo, Japan} \hfill        April 2015 - January 2019
\vspace{8pt}

\title{APPLICATION ENGINEER}
\employer{}
\location{}
\dates{January 2018 - January 2019}
\begin{position}
Worked on the frontend portal for our self-service environment.
Originally developed as a collection of React components, I introduced
many libraries to architect and test the application.\\

 \begin{itemize} \itemsep -2pt % reduce space between items
   \item Utilize Redux to manage global application state, write custom
middlewares accomplish side effects such as API calls and saving data
to local storage, author selectors using reselect and re-reselect.
  \item Lint the codebase using eslint and the airbnb rules.
  \item Author tests based on mocha, chai, sinon, enzyme.
  \item Introduce react-final-form to manage the form state.
  \item Add PostCSS and integrate it with webpack to build our CSS
resources which included Bootstrap, Font Awesome and others.
  \item Minify images using webpack and the imagemin plugin.
 \end{itemize}
\end{position}

\title{DEVOPS ENGINEER}
\employer{}
\location{}
\dates{June 2015  - December 2017}
\begin{position}
Originally assigned to the Server Platform Team upon joining the company.
Helped maintain our many storage solutions (Pure Storage, Nimble Storage, etc),
our OpenStack environment, our SDN environment, and others.
Some of the tasks I accomplished include: \\

\begin{itemize} \itemsep -2pt % reduce space between items
  \item Create a dashboard which visualizes storage telemetry such as operation
rates and latency. This was built using Fluentd, InfluxDB, Grafana, SNMP,
Paramiko and custom Python-based tools.
  \item Author Python scripts for automation of routine operations of our
SDN environment. These scripts simplified and increased the reliability
of routine operations such as backups and resource allocation with the
hosting VMware cluster. Made use of pyVmomi to manage VMware resources.
  \item Fix a problem in our OpenStack environment where sysctl variables would
be reset upon restart. Update the Ansible playbooks and provisioned
the fixes to all the relevant nodes.
  \item Migrate OpenStack instances from one of our tenants to newly added
tenants. This involved researching and testing the procedure which
included manual updates of the relevant MySQL records, writing a manual
and executing the operation successfully.
\end{itemize}
\end{position}

\vspace{0.2in} 
\section{\centerline{EDUCATION}}
\vspace{8pt} 
{\sl Bachelor of Science}, Physics \\ % \sl will be bold italic in
Osaka University      \hfill    March, 2015
  
\vspace{0.2in} 

\section{\centerline{COURSES}}
\vspace{8pt} 

\begin{tabular}{@{}ll}
  M101: MongoDB for Developers & \href{http://university.mongodb.com/course_completion/066efa7d13f84df789a4ddff81524640}{Certificate} \\
  M102: MongoDB for DBAs & \href{http://university.mongodb.com/course_completion/797324e4c82e4a27b2d33c3749476003}{Certificate} \\
  Kubernetes Certification by Mirantis (KCM100) & \href{https://training.mirantis.com/verify/certificate/status/bF19JZmlQoYyAoU4jkY15w/100-457-704}{Certificate} \\
\end{tabular}

\newpage

\section{\centerline{OPEN SOURCE CONTRIBUTIONS}}
\vspace{8pt} 

Made several contributions to open source communities, mostly
centered around JavaScript and Node.js, which can be found publically
on GitHub.

\subsubsection{PULL REQUESTS}

\begin{itemize}
  \item \url{https://github.com/socketio/socket.io/pull/2745} \\
  Socket.IO is a popular websocket library for Node.js which provides
  abstractions for different kinds of persistent connections, such as web
  sockets and XHR polling, and additional features such as multiplexing.
  This PR fixed a bug introduced in version 1.5.0 of the library, where
  some query string parameters were being ignored. The solution was to use
  the ES5 Object.assign function. A polyfill to the function was added in
  order to keep backwards compatibility with old IE browsers which were
  still being supported at the time.

  \item \url{https://github.com/socketio/engine.io/pull/444} \\
  This PR fixes a bug introduced in version 1.5.1 of the library.
  While the fix itself was a simple one line removal, I introduced a
  regression test to ensure that the bug does not resurface in the future.

  \item \url{https://github.com/reduxjs/react-redux/pull/1288} \\
  After a lengthy discussion on the right approach to implement component rerenders
  on the new react-redux hooks based API, we settled to use referential equality, as
  opposed to shallow equality, which was used in the HoC based approach. This mirrors
  the way that official react hooks, such as useState, handle rerenders.

  \item \url{https://github.com/agentk/fontfacegen/pull/26} \\
  fontfacegen is a library that uses FontForge in the background to generate
  all the font formats required to use custom fonts on the browser. This PR
  introduces a new feature to be able to select only a subset of the
  characters from the font files to be output. This is especially important
  in font sets for Japanese, where the whole font file size could be multiple
  MB, but in reality only a handful of characters are needed, reducing the
  size significantly.

  \item \url{https://github.com/libxmljs/libxmljs/pull/521} \\
  Adds support for XML processing instructions to the libxmljs library.
  One use case is for adding a stylesheet to svg files.
  libxmljs is a wrapper around the C library libxml2, and therefore this
  PR is mostly C++ code which binds the Node.JS JavaScript engine, V8, to
  the relevant libxml2 functions. Implementing this PR was very instructive
  in regards to the inner workings of Node.JS.

  \item \url{https://github.com/final-form/react-final-form/pull/266} \\
  Very useful library to manage form state, based on the observer pattern.
  This PR introduces an example demonstrating how to integrate lookahead components
  with react-final-form and redux, and demonstrates the usage of different
  features of the library.

  \item \url{https://github.com/jasonpincin/stream-to-string/pull/1} \\
  This small module converts a node stream into a string.
  Adds the option to specify a specific encoding when converting a stream to
  a string, making it a non-breaking change and adding a regression test.
  This PR is later used in my own vinyl-to-string NPM module.

  \item \url{https://github.com/izolate/html2pug/pull/2} \\
  Converts HTML into Pug, an alternative and less verbose templating language,
  popular to render HTML in Express applications. The main purpose of this PR
  is to replace the jsdom as the HTML parser with the standard parse5 HTML
  parser, allowing to handle features of HTML such as comments, parse fragments,
  etc.

  \item \url{https://github.com/itgalaxy/favicons/pull/110} \\
  Small NPM module used to generate favicons for different platforms and their
  associated files. This PR introduced the option to add the generated HTML
  needed to embed to a webpage to use the generated favicons, to the files piped
  in the gulp plugin version of the module. This enables additional modifications
  to the HTML, such as converting it to Pug using gulp-html2pug.
\end{itemize}

\subsubsection{NPM MODULES}

\begin{itemize}
  \item \url{https://www.npmjs.com/package/i18next-conv} \\
  Due to my many contributions to i18next, I am a member of the official
  GitHub i18next organization and became the main maintainer of this module.
  This module parses and converts PO files into the i18next format which
  is based on JSON. This is useful for teams where the translators use
  the gettext industry standard and an integration with i18next is needed.
  It has over 5000 weekly downloads.

  \item \url{https://www.npmjs.com/package/i18next-fetch-backend} \\
  One of my contributions to the extensive and popular JavaScript i18n
  environment, i18next, this library provides an alternative backend
  using fetch, to the default XHR backend, which is useful in environments
  where i18next is used both in the server and in the browser.
  Currently it has over 2000 monthly downloads.

  \item \url{https://www.npmjs.com/package/react-scroll-ondrag} \\
  React library which exposes a hook to add scrolling ability to react
  elements by dragging the mouse. Provides a general and easy-to-implement
  solution to the problem.

  \item \url{https://www.npmjs.com/package/react-stay-scrolled} \\
  Currently having 34 stars in GitHub and over 1000 monthly downloads,
  this React library helps containers such as chat windows scroll down
  automatically whenever new messages arrive.

  \item \url{https://www.npmjs.com/package/vinyl-contents-tostring} \\
  Module used to convert the content of vinyl files (which are the backbone
  of gulp) into strings. This is used in many of my gulp plugins.
  This module has over 1000 monthly downloads.

  \item \url{https://www.npmjs.com/package/rollup-plugin-i18next-conv} \\
  Rollup plugin integrating i18next-conv. Enables import of PO files,
  converting them into JSON resources, which can then be handed to
  i18next.
  This is useful especially when building a webview application, where
  all the content has to be bundled into the app to be used offline.

  \item \url{https://www.npmjs.com/package/gulp-xml-transformer} \\
  Gulp integration for libxmljs.
  This module has over 1000 monthly downloads.
\end{itemize}

\vspace{0.2in} 
\section{\centerline{LANGUAGES}} 
\vspace{15pt}
\begin{itemize}
   \item Native Spanish
   \item Fluent English
   \item Fluent Hebrew
   \item Japanese Business (JLPT N1 Certified)
\end{itemize}


\vspace{0.2in} 
 
\end{resume} 
\end{document}
